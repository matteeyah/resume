%%%%%%%%%%%%%%%%%%%%%%%%%%%%%%%%%%%%%%%
% Deedy - One Page Two Column Resume
% LaTeX Template
% Version 1.2 (16/9/2014)
%
% Original author:
% Debarghya Das (http://debarghyadas.com)
%
% Original repository:
% https://github.com/deedydas/Deedy-Resume
%
% IMPORTANT: THIS TEMPLATE NEEDS TO BE COMPILED WITH XeLaTeX
%
% This template uses several fonts not included with Windows/Linux by
% default. If you get compilation errors saying a font is missing, find the line
% on which the font is used and either change it to a font included with your
% operating system or comment the line out to use the default font.
%
%%%%%%%%%%%%%%%%%%%%%%%%%%%%%%%%%%%%%%


\documentclass[]{matija-resume}
\usepackage{fancyhdr}
\usepackage{fontawesome}

\pagestyle{fancy}
\fancyhf{}

\begin{document}

%%%%%%%%%%%%%%%%%%%%%%%%%%%%%%%%%%%%%%
%
%     LAST UPDATED DATE
%
%%%%%%%%%%%%%%%%%%%%%%%%%%%%%%%%%%%%%%
\lastupdated

%%%%%%%%%%%%%%%%%%%%%%%%%%%%%%%%%%%%%%
%
%     TITLE NAME
%
%%%%%%%%%%%%%%%%%%%%%%%%%%%%%%%%%%%%%%
\namesection{Matija}{Cupić} {
\urlstyle{same}\href{https://matteeyah.com}{matteeyah.com} |
\href{mailto:matteeyah@gmail.com}{matteeyah@gmail.com} \\
\vspace{\topsep}
GitHub:// \href{https://github.com/matteeyah}{\bf matteeyah \faExternalLink} |
GitLab:// \href{https://gitlab.com/matteeyah}{\bf matteeyah \faExternalLink} |
LinkedIn://  \href{https://www.linkedin.com/in/matteeyah}{\bf matteeyah \faExternalLink} |
StackOverflow://  \href{https://stackoverflow.com/users/1139722/matteeyah}{\bf matteeyah \faExternalLink}
}

\begin{minipage}[t]{1.0\textwidth}

%%%%%%%%%%%%%%%%%%%%%%%%%%%%%%%%%%%%%%
%     EXPERIENCE
%%%%%%%%%%%%%%%%%%%%%%%%%%%%%%%%%%%%%%

\section{Experience}

\runsubsection{OpTonal}
\descript{| Co-Founder | CTO }
\location{June 2023 - Present | Remote}
\vspace{\topsep}
Due to a weird turn of events one of the original co-founders left, and I stepped up and "became" a co-founder. In my "new" role I continued delivering daily product changes and meeting regularly with customers.\\
\vspace{\topsep}
A new responsibility was making strategic technical decisions. We've "bet the business" on the direction of AI development and focusing on using publicly available general models instead of investing in developing custom specialized models. This allowed us to pull ahead of the competition by focusing on features that are not made redundant by the rapid advancement of AI.\\
\vspace{\topsep}
Another large aspect of my work was building on top of what I built previously and creating a culture of "breakneck" engineering velocity. We set 30 changes / engineer / month as an ambitious goal. I focused on finding ways to empower engineers to make decisions, be autonomous and deliver changes as independently as possible, while still providing support when needed.\\
\vspace{\topsep}
I spent lots of engineering effort on resolving tech debt and adopting idiomatic Rails patterns. This allowed our engineers to work with the least amount of overhead possible.
\vspace{\topsep}

\sectionsep

\runsubsection{OpTonal}
\descript{| Founding Engineer }
\location{October 2022 - June 2023 | Remote}
\vspace{\topsep}
I was the first engineering hire at OpTonal. After joining, my first priority was to transition product development from an outsourced effort to an in-house effort. This allowed us to fully control the quality and velocity of our product development.\\
\vspace{\topsep}
Building on top of the "quality and velocity" theme, the next priority was creating a safe environment that nurtures enthusiastic and fast changes. I worked on making sure everyone feels comfortable with making frequent and far-reaching changes by creating systems and processes that act as a safety net.\\
\vspace{\topsep}
Over time, I lead the transition from using an external agency to having a team of 7 engineers in-house that I managed. I supported the transition by researching and drafting job postings, interviewing applicants and building info packages for executives about applicants.\\
\vspace{\topsep}
During and after this transition period I delivered numerous product and architecture improvements and met with clients frequently.
\vspace{\topsep}

\sectionsep

\runsubsection{Pennylane}
\descript{| Team Lead }
\location{May 2022 - October 2022 | Remote}
\vspace{\topsep}
I was part of the banking integrations team that focused on integrating with various traditional and neo banking providers.\\
\vspace{\topsep}
I managed a team of 6 people and worked cross-functionally with several adjacent teams to deliver stability for mission critical systems and make banking integrations operate smoothly.\\
\vspace{\topsep}
A big focus area for me was implementing a system for measuring and improving observability for banking systems stability.
\sectionsep

\runsubsection{GitLab}
\descript{| Contributor Success Fullstack Engineer }
\location{December 2021 - May 2022 | Remote}
\vspace{\topsep}
I joined the Contributor Success team to improve processes for everyone contributing to GitLab and support decision makers in guiding the future of the contribution journey at GitLab.\\
\vspace{\topsep}
Here's some examples of the work that I did in the Contributor Success team:\\
\vspace{\topsep}
\begin{tightemize}
\item Equipping engineering managers to better handle community contributions
\item Empowering engineers to be more efficient in their work by reducing the development process overhead
\item Creating Business Intelligence tools that decision makers use to make data driven decisions about engineering goals
\end{tightemize}
\vspace{\topsep}
Before leaving I onboarded a Director of Contributor Success to continue driving improvements for the community.
\sectionsep

\end{minipage}

\begin{minipage}[t]{1.0\textwidth}

\runsubsection{GitLab}
\descript{| Backend Engineer }
\location{October 2018 - December 2021 | Remote}
\vspace{\topsep}
While I was part of the CI team we worked on building and scaling GitLab's CI system for hundreds of millions of daily active users. The system we built now serves \textasciitilde 1B CI builds every \textasciitilde 9 months.\\
\vspace{\topsep}
Over time I became a reviewer and maintainer for various GitLab components.
\vspace{\topsep}
\begin{tightemize}
\item \href{https://gitlab.com/gitlab-org/gitlab-development-kit/}{Maintainer @ GitLab Development Kit \faicon{external-link}}
\item \href{https://gitlab.com/gitlab-org/gitlab/-/tree/master/lib/gitlab/ci/templates}{Maintainer @ GitLab CI/CD Templates \faicon{external-link}}
\item \href{https://gitlab.com/gitlab-org/gitlab/}{Reviewer @ GitLab \faicon{external-link}}
\item \href{https://about.gitlab.com/job-families/expert/merge-request-coach/}{MR Coach \faicon{external-link}}
\end{tightemize}

\vspace{\topsep}
I drove and implemented over 400 deliverables in the main web app and surrounding micro-services. Some notable work includes:
\vspace{\topsep}
\begin{tightemize}
\item Support for multiple Kubernetes clusters per project | \href{https://gitlab.com/gitlab-org/gitlab/-/issues/3734}{Issue \faicon{external-link}}
\item Mechanism for retaining latest artifact per ref indefinitely | \href{https://gitlab.com/gitlab-org/gitlab/-/issues/16267}{Issue \faicon{external-link}} | \href{https://gitlab.com/gitlab-org/gitlab/-/merge_requests/29802}{Merge Request \#1 \faicon{external-link}} | \href{https://gitlab.com/gitlab-org/gitlab/-/merge_requests/30741}{Merge Request \#2 \faicon{external-link}}
\item Project pipeline subscriptions | \href{https://gitlab.com/gitlab-org/gitlab/-/issues/9045}{Issue \faicon{external-link}} | \href{https://gitlab.com/gitlab-org/gitlab/-/merge_requests/20063}{Merge Request \faicon{external-link}}
\item Specifying variables when running manual jobs | \href{https://gitlab.com/gitlab-org/gitlab-foss/-/issues/24935}{Issue \faicon{external-link}} | \href{https://gitlab.com/gitlab-org/gitlab-foss/-/merge_requests/30485}{Merge Request \faicon{external-link}}
\item Protected variable log masking | \href{https://gitlab.com/gitlab-org/gitlab-foss/-/issues/13784}{Issue \faicon{external-link}} | \href{hhttps://gitlab.com/gitlab-org/gitlab-foss/-/merge_requests/25293}{Merge Request \faicon{external-link}}
\item CI job parallelization | \href{https://gitlab.com/gitlab-org/gitlab-foss/-/issues/21480}{Issue \faicon{external-link}} | \href{https://gitlab.com/gitlab-org/gitlab-foss/-/merge_requests/22631}{Merge Request \faicon{external-link}}
\item Status checking for pipeline triggers | \href{https://gitlab.com/gitlab-org/gitlab/-/issues/11238}{Issue \faicon{external-link}} | \href{https://gitlab.com/gitlab-org/gitlab/-/merge_requests/15580}{Merge Request \faicon{external-link}}
\end{tightemize}
\sectionsep

\runsubsection{GitLab}
\descript{| Community Advocate Manager }
\location{April 2018 – October 2018 | Remote}
\begin{tightemize}
\item Implemented the GitLab Ultimate for Education and Open Source programs
\item Hired and onboarded a whole new Community Advocacy team
\end{tightemize}
\sectionsep

\runsubsection{GitLab}
\descript{| Community Advocate }
\location{November 2016 – April 2018 | Remote}
\vspace{\topsep}
I started working at GitLab with a mission to cultivate and reshape GitLab's online community.
\vspace{\topsep}
\begin{tightemize}
\item Implemented custom middleware that integrates Disqus and HackerNews into Zendesk
\item Migrated the Community Advocacy team from a shared Zendesk instance
\item Reduced friction in the Community Advocacy workflow
\item Brought all response time averages below 6h
\end{tightemize}
\sectionsep

\end{minipage}
%%%%%%%%%%%%%%%%%%%%%%%%%%%%%%%%%%%%%%
%     PROJECTS & AWARDS
%%%%%%%%%%%%%%%%%%%%%%%%%%%%%%%%%%%%%%

\begin{minipage}[t]{1.0\textwidth}

\section{Projects \& Awards}
\begin{tabular}{rll}
\href{https://github.com/matteeyah/respondo}{GitHub Project \faExternalLink} & Respondo & Ticketing system for social media posts\\
\href{https://handbook.gitlab.com/handbook/engineering/projects/#tanukidesk}{GitLab Page \faExternalLink} & Tanukidesk & Zendesk connector for Disqus/HackerNews\\
\href{https://github.com/matteeyah/resume}{GitHub Project \faExternalLink} & This Resume & Open-source and written in LaTeX\\
&&\\
2016 & 1\textsuperscript{st} / Country & IT Project of the Year\\
2016 & Gold / World & International Conference of Young Scientists\\
2016 & 1\textsuperscript{st} / Country & National conference of research projects
\end{tabular}
\sectionsep

%%%%%%%%%%%%%%%%%%%%%%%%%%%%%%%%%%%%%%
%     PUBLICATIONS
%%%%%%%%%%%%%%%%%%%%%%%%%%%%%%%%%%%%%%

\section{Publications}
\renewcommand\refname{\vskip -1.5em} % Couldn't get this working from the .cls file
\bibliographystyle{abbrv}
\bibliography{publications}
\nocite{*}

\end{minipage}
\end{document}  \documentclass[]{article}
