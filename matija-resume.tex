%%%%%%%%%%%%%%%%%%%%%%%%%%%%%%%%%%%%%%%
% Deedy - One Page Two Column Resume
% LaTeX Template
% Version 1.2 (16/9/2014)
%
% Original author:
% Debarghya Das (http://debarghyadas.com)
%
% Original repository:
% https://github.com/deedydas/Deedy-Resume
%
% IMPORTANT: THIS TEMPLATE NEEDS TO BE COMPILED WITH XeLaTeX
%
% This template uses several fonts not included with Windows/Linux by
% default. If you get compilation errors saying a font is missing, find the line
% on which the font is used and either change it to a font included with your
% operating system or comment the line out to use the default font.
%
%%%%%%%%%%%%%%%%%%%%%%%%%%%%%%%%%%%%%%


\documentclass[]{matija-resume}
\usepackage{fancyhdr}
\usepackage{fontawesome}

\pagestyle{fancy}
\fancyhf{}

\begin{document}

%%%%%%%%%%%%%%%%%%%%%%%%%%%%%%%%%%%%%%
%
%     LAST UPDATED DATE
%
%%%%%%%%%%%%%%%%%%%%%%%%%%%%%%%%%%%%%%
\lastupdated

%%%%%%%%%%%%%%%%%%%%%%%%%%%%%%%%%%%%%%
%
%     TITLE NAME
%
%%%%%%%%%%%%%%%%%%%%%%%%%%%%%%%%%%%%%%
\namesection{Matija}{Cupić} {
\urlstyle{same}\href{https://matteeyah.com}{matteeyah.com} |
\href{mailto:matteeyah@gmail.com}{matteeyah@gmail.com} \\
\vspace{\topsep}
GitHub:// \href{https://github.com/matteeyah}{\bf matteeyah \faExternalLink} |
GitLab:// \href{https://gitlab.com/matteeyah}{\bf matteeyah \faExternalLink} |
LinkedIn://  \href{https://www.linkedin.com/in/matteeyah}{\bf matteeyah \faExternalLink} |
StackOverflow://  \href{https://stackoverflow.com/users/1139722/matteeyah}{\bf matteeyah \faExternalLink}
}

\begin{minipage}[t]{1.0\textwidth}

%%%%%%%%%%%%%%%%%%%%%%%%%%%%%%%%%%%%%%
%     EXPERIENCE
%%%%%%%%%%%%%%%%%%%%%%%%%%%%%%%%%%%%%%

\section{Experience}

\runsubsection{OpTonal}
\descript{| Co-Founder | CTO }
\location{June 2023 - Present | Remote}
\vspace{\topsep}
Due to a weird turn of events one of the two OpTonal co-founders left. I stepped up and "became" a co-founder. I continued delivering daily product changes and meeting with customers.\\
\vspace{\topsep}
A new responsibility was making strategic technical decisions. I've "bet the business" on the direction of AI development. We focused on using publicly available general models instead of  developing custom models. This allowed us to pull ahead of the competition. I prioritized features that are not made redundant by the rapid advancement of AI.\\
\vspace{\topsep}
My other focus was doubling down on creating a culture of "breakneck" engineering velocity. We set 30 changes / engineer / month as an ambitious goal. I searched for ways to empower engineers to make decisions, be autonomous and deliver changes as independently as possible.\\
\vspace{\topsep}
I resolved tech debt and adopted idiomatic Rails patterns. This allowed our engineers to work with the least amount of overhead possible.
\vspace{\topsep}

\sectionsep

\runsubsection{OpTonal}
\descript{| Founding Engineer }
\location{October 2022 - June 2023 | Remote}
\vspace{\topsep}
I was the first engineering hire at OpTonal. My first priority was transitioning ownership of product development from an outsourcing agency. This allowed us to own our engineering quality and velocity. Over time, we moved from using an external agency to having a team of 7 engineers that I managed. I enabled the transition by researching and drafting job listings and interviewing applicants.\\
\vspace{\topsep}
After assuming ownership of engineering, I set out to set a high bar for both quality and velocity. I built an environment that facilitates enthusiastic and fast changes. A key part was making everyone feel comfortable with making frequent and far-reaching changes. I built systems and processes that acted as a "safety net" for the engineers.\\
\vspace{\topsep}
Throughout my time as a founding engineer I also:\\
\vspace{\topsep}
\begin{tightemize}
\item Delivered many product improvements
\item Drove architecture improvements
\item Met with clients often to align engineering with product needs
\end{tightemize}

\sectionsep

\runsubsection{Pennylane}
\descript{| Team Lead }
\location{May 2022 - October 2022 | Remote}
\vspace{\topsep}
I was part of the banking integrations team at Pennylane. We focused on integrating with various traditional and neo banking providers.\\
\vspace{\topsep}
I managed a team of 6 people and worked with several adjacent teams. My biggest priority was providing stability for mission critical systems. I lead the implementation of observability across the banking integrations. This allowed us to track stability across legacy and new systems.\\
\vspace{\topsep}
The integrations didn't have an overarching architecture and differed in quality and design. I lead the overhauling of these systems. This allowed any engineer on the team to work on any integration. Which lead to a reduction of knowledge siloing and made scheduling changes easier.\\
\sectionsep

\end{minipage}

\footnote{This Resume is open-source and written in LaTeX - \href{https://github.com/matteeyah/resume}{GitHub Project \faExternalLink}}

\begin{minipage}[t]{1.0\textwidth}

\runsubsection{GitLab}
\descript{| Contributor Success Fullstack Engineer }
\location{December 2021 - May 2022 | Remote}
\vspace{\topsep}
I joined the Contributor Success team at GitLab because of its mission. Improving processes for everyone contributing to GitLab. I was the first member of the team and laid the foundation for what the team would grow into today.\\
\vspace{\topsep}
While at the Contributor Success team I:\\
\vspace{\topsep}
\begin{tightemize}
\item Systematized community contributions. Created tools that create visibility into these contributions. This empowered engineering manager to process them easier and faster.
\item Reduced development process overhead, which made engineers more efficient
\item Created Business Intelligence tools for engineering productivity. This allowed decision makers to data driven decisions about engineering at GitLab
\end{tightemize}
\vspace{\topsep}
Before leaving, I onboarded a Director of Contributor Success. He continued building on the improvements that I introduced.
\sectionsep

\runsubsection{GitLab}
\descript{| Backend Engineer }
\location{October 2018 - December 2021 | Remote}
\vspace{\topsep}
I joined the GitLab CI team at the invite of the team's engineering manager. This was a part of an inner-sourcing campaign.\\
\vspace{\topsep}
At the GitLab CI team I gained a lot of experience working with large scale distributed systems. We worked on building and scaling the CI system for hundreds of millions of daily active users.\\
\vspace{\topsep}
The system we built now serves \textasciitilde 1B CI builds every \textasciitilde 9 months.\\
\vspace{\topsep}
Over time I became a reviewer and maintainer for various GitLab components.
\vspace{\topsep}
\begin{tightemize}
\item \href{https://gitlab.com/gitlab-org/gitlab-development-kit/}{Maintainer @ GitLab Development Kit \faicon{external-link}}
\item \href{https://gitlab.com/gitlab-org/gitlab/-/tree/master/lib/gitlab/ci/templates}{Maintainer @ GitLab CI/CD Templates \faicon{external-link}}
\item \href{https://gitlab.com/gitlab-org/gitlab/}{Reviewer @ GitLab \faicon{external-link}}
\item \href{https://about.gitlab.com/job-families/expert/merge-request-coach/}{MR Coach \faicon{external-link}}
\end{tightemize}
\vspace{\topsep}
I implemented over 400 deliverables in the main web app and surrounding micro-services. Most of my work is \href{https://gitlab.com/gitlab-org/gitlab/-/merge_requests/?sort=updated_desc&state=merged&author_username=matteeyah&first_page_size=20}{publicly \faicon{external-link}} \href{https://gitlab.com/gitlab-org/gitlab-foss/-/merge_requests?scope=all&state=merged&author_username=matteeyah}{available \faicon{external-link}}.\\
\vspace{\topsep}
\sectionsep

\runsubsection{GitLab}
\descript{| Community Advocate Manager }
\location{April 2018 – October 2018 | Remote}
\vspace{\topsep}
The Community Advocacy team that I was a part of got restructured, and I was tasked with rebuilding it. It was my first time managing people. The goal was creating a team that maintains GitLab's online presence. It was critical to make sure everyone was aligned with the company culture.\\
\vspace{\topsep}
I focused on rehiring for the team and onboarding new members.  I ended up sourcing, interviewing, hiring and onboarding 4 new team members.\\
\vspace{\topsep}
Apart from maintaining our online presence, I also focused on finding ways to give back to the community. I came up with the GitLab Ultimate for Education and Open Source programs. They offer a free subscription for students and qualifying open source projects.
\sectionsep

\runsubsection{GitLab}
\descript{| Community Advocate }
\location{November 2016 – April 2018 | Remote}
\vspace{\topsep}
I started working at GitLab with a mission to cultivate and reshape GitLab's online community.
\vspace{\topsep}
\begin{tightemize}
\item Implemented custom middleware that integrates Disqus and HackerNews into Zendesk
\item Started and maintained a team Zendesk instance for tracking social posts
\item Brought response time averages for all social posts to below 6h
\item Delivered many Community Advocacy workflow improvements
\end{tightemize}
\sectionsep

\end{minipage}
%%%%%%%%%%%%%%%%%%%%%%%%%%%%%%%%%%%%%%
%     PROJECTS & AWARDS
%%%%%%%%%%%%%%%%%%%%%%%%%%%%%%%%%%%%%%

\begin{minipage}[t]{1.0\textwidth}

\section{Awards}
\begin{tabular}{rll}
2016 & 1\textsuperscript{st} / Country & IT Project of the Year\\
2016 & Gold / World & International Conference of Young Scientists\\
2016 & 1\textsuperscript{st} / Country & National conference of research projects
\end{tabular}
\sectionsep
\end{minipage}
\end{document}  \documentclass[]{article}
