%%%%%%%%%%%%%%%%%%%%%%%%%%%%%%%%%%%%%%%
% Deedy - One Page Two Column Resume
% LaTeX Template
% Version 1.2 (16/9/2014)
%
% Original author:
% Debarghya Das (http://debarghyadas.com)
%
% Original repository:
% https://github.com/deedydas/Deedy-Resume
%
% IMPORTANT: THIS TEMPLATE NEEDS TO BE COMPILED WITH XeLaTeX
%
% This template uses several fonts not included with Windows/Linux by
% default. If you get compilation errors saying a font is missing, find the line
% on which the font is used and either change it to a font included with your
% operating system or comment the line out to use the default font.
% 
%%%%%%%%%%%%%%%%%%%%%%%%%%%%%%%%%%%%%%

\documentclass[]{matija-resume}
\usepackage{fancyhdr}
 
\pagestyle{fancy}
\fancyhf{}
 
\begin{document}

%%%%%%%%%%%%%%%%%%%%%%%%%%%%%%%%%%%%%%
%
%     LAST UPDATED DATE
%
%%%%%%%%%%%%%%%%%%%%%%%%%%%%%%%%%%%%%%
\lastupdated

%%%%%%%%%%%%%%%%%%%%%%%%%%%%%%%%%%%%%%
%
%     TITLE NAME
%
%%%%%%%%%%%%%%%%%%%%%%%%%%%%%%%%%%%%%%
\namesection{Matija}{Ćupić} {
\urlstyle{same}\href{https://matteeyah.com}{matteeyah.com} |
\href{mailto:matteeyah@gmail.com}{matteeyah@gmail.com}
}

%%%%%%%%%%%%%%%%%%%%%%%%%%%%%%%%%%%%%%
%
%     COLUMN ONE
%
%%%%%%%%%%%%%%%%%%%%%%%%%%%%%%%%%%%%%%

\begin{minipage}[t]{0.33\textwidth} 


%%%%%%%%%%%%%%%%%%%%%%%%%%%%%%%%%%%%%%
%     LINKS
%%%%%%%%%%%%%%%%%%%%%%%%%%%%%%%%%%%%%%

\section{Links} 
GitHub:// \href{https://github.com/matteeyah}{\bf matteeyah} \\
LinkedIn://  \href{https://www.linkedin.com/in/matteeyah}{\bf matteeyah} \\
StackOverflow://  \href{https://stackoverflow.com/users/1139722/matteeyah}{\bf matteeyah}

%%%%%%%%%%%%%%%%%%%%%%%%%%%%%%%%%%%%%%
%     Projects
%%%%%%%%%%%%%%%%%%%%%%%%%%%%%%%%%%%%%%

\section{Projects}
\subsection{Tanukidesk}
\location{\href{https://gitlab.com/gitlab-com/marketing/community-relations/community-advocacy/tanukidesk}{GitLab Project}}
Bidirectional communication between Zendesk and Disqus / HackerNews.
\sectionsep
\subsection{Respondo}
\location{\href{https://respondohub.com/}{Website}}
Respondo is a ticketing system for social media posts.
\sectionsep

%%%%%%%%%%%%%%%%%%%%%%%%%%%%%%%%%%%%%%
%     EDUCATION
%%%%%%%%%%%%%%%%%%%%%%%%%%%%%%%%%%%%%%

\section{Education} 

\subsection{Metropolitan University}
\descript{BS in Software Engineering}
\location{October 2016 - | Belgrade, Serbia}
Faculty of Information Technology \\
\sectionsep

\subsection{Electrotechnical School}
\location{Grad. June 2016 | Belgrade, Serbia}
\sectionsep

%%%%%%%%%%%%%%%%%%%%%%%%%%%%%%%%%%%%%%
%     COURSEWORK
%%%%%%%%%%%%%%%%%%%%%%%%%%%%%%%%%%%%%%

\section{Coursework}
\subsection{Undergraduate}
Software Engineering \\
Systems Engineering \\
Hardware Systems \\
Distributed Systems \\
Concurrent Programming \\
Advanced Math \\
Machine Learning \\
\sectionsep

%%%%%%%%%%%%%%%%%%%%%%%%%%%%%%%%%%%%%%
%
%     COLUMN TWO
%
%%%%%%%%%%%%%%%%%%%%%%%%%%%%%%%%%%%%%%

\end{minipage} 
\hfill
\begin{minipage}[t]{0.66\textwidth} 

%%%%%%%%%%%%%%%%%%%%%%%%%%%%%%%%%%%%%%
%     EXPERIENCE
%%%%%%%%%%%%%%%%%%%%%%%%%%%%%%%%%%%%%%

\section{Experience}
\runsubsection{GitLab}
\descript{| Backend Engineer }
\location{October 2018 - Present | Remote}
\vspace{\topsep}
After bootstrapping the Community Advocacy team I went back to my biggest passion at work - engineering software. I joined the CI/CD Engineering team after the 2017 GitLab Greece Summit challenge. The challenge was to develop a proof of concept Web IDE. I implemented the backend that spins up kubernetes pods for development and application live preview and exposes their underlying file system as well as a port to access the preview application at. I work as a Backend Engineer on the CI team to make sure our CI offering remains the best on the market.
\sectionsep

\runsubsection{GitLab}
\descript{| Community Advocate Manager }
\location{April 2018 – October 2018 | Remote}
\vspace{\topsep}
As luck would have it I got the the opportunity to manage the Community Advocacy team at GitLab in May 2018. I always liked helping others do the best work they can so this was a no-brainer. As a Community Advocate Manager I built and cultivated a new advocacy team.
\vspace{\topsep}
\begin{tightemize}
\item As a follow-up from the \#movingtogitlab initiative, I implemented the GitLab Ultimate for Education and Open Source programs.
\item Hired and onboarded a new Community Advocacy team.
\item Overhauled GitLab's Community Advocacy handbook and documented the Community Advocacy process at GitLab
\item Reworked the Community Advocacy process at GitLab to be more transparent, iterative and collaborative.
\end{tightemize}
\sectionsep

\runsubsection{GitLab}
\descript{| Community Advocate }
\location{November 2016 – April 2018 | Remote}
\vspace{\topsep}
I started working as a Community Advocate at GitLab with a mission to cultivate and reshape GitLab's online community.
\vspace{\topsep}
\begin{tightemize}
\item Implemented custom middleware that integrates Disqus and HackerNews into Zendesk - Tanukidesk.
\item Successfully migrated the Community Advocacy team from the main support Zendesk instance and to a separate Community Advocacy Zendesk instance and administered it.
\item  Improved the Community Advocacy workflow and brought all response time averages below 6h (except Hacker News which had a response time of less than 1h).
\end{tightemize}
\sectionsep

%%%%%%%%%%%%%%%%%%%%%%%%%%%%%%%%%%%%%%
%     AWARDS
%%%%%%%%%%%%%%%%%%%%%%%%%%%%%%%%%%%%%%

\section{Awards (Computer Science)} 
\begin{tabular}{rll}
2016 & 1\textsuperscript{st} / Country & IT Project of the Year\\
2016 & Gold Medal / World & International Conference of Young Scientists\\
2016 & 1\textsuperscript{st} / Country & National conference of research projects\\
2015 & 3\textsuperscript{rd} / Country & National conference of research projects\\
2014 & 1\textsuperscript{st} / Country & National conference of research projects\\
2014 & 3\textsuperscript{rd} / Country & National conference of research project
\end{tabular}
\sectionsep

%%%%%%%%%%%%%%%%%%%%%%%%%%%%%%%%%%%%%%
%     PUBLICATIONS
%%%%%%%%%%%%%%%%%%%%%%%%%%%%%%%%%%%%%%

\section{Publications} 
\renewcommand\refname{\vskip -1.5em} % Couldn't get this working from the .cls file
\bibliographystyle{abbrv}
\bibliography{publications}
\nocite{*}

\end{minipage} 
\end{document}  \documentclass[]{article}
