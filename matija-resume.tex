%%%%%%%%%%%%%%%%%%%%%%%%%%%%%%%%%%%%%%%
% Deedy - One Page Two Column Resume
% LaTeX Template
% Version 1.2 (16/9/2014)
%
% Original author:
% Debarghya Das (http://debarghyadas.com)
%
% Original repository:
% https://github.com/deedydas/Deedy-Resume
%
% IMPORTANT: THIS TEMPLATE NEEDS TO BE COMPILED WITH XeLaTeX
%
% This template uses several fonts not included with Windows/Linux by
% default. If you get compilation errors saying a font is missing, find the line
% on which the font is used and either change it to a font included with your
% operating system or comment the line out to use the default font.
%
%%%%%%%%%%%%%%%%%%%%%%%%%%%%%%%%%%%%%%


\documentclass[]{matija-resume}
\usepackage{fancyhdr}
\usepackage{fontawesome}

\pagestyle{fancy}
\fancyhf{}

\begin{document}

%%%%%%%%%%%%%%%%%%%%%%%%%%%%%%%%%%%%%%
%
%     LAST UPDATED DATE
%
%%%%%%%%%%%%%%%%%%%%%%%%%%%%%%%%%%%%%%
\lastupdated

%%%%%%%%%%%%%%%%%%%%%%%%%%%%%%%%%%%%%%
%
%     TITLE NAME
%
%%%%%%%%%%%%%%%%%%%%%%%%%%%%%%%%%%%%%%
\namesection{Matija}{Cupic} {
\urlstyle{same}\href{https://matteeyah.com}{matteeyah.com} |
\href{mailto:matteeyah@gmail.com}{matteeyah@gmail.com}
}

%%%%%%%%%%%%%%%%%%%%%%%%%%%%%%%%%%%%%%
%
%     COLUMN ONE
%
%%%%%%%%%%%%%%%%%%%%%%%%%%%%%%%%%%%%%%

\begin{minipage}[t]{0.33\textwidth}


%%%%%%%%%%%%%%%%%%%%%%%%%%%%%%%%%%%%%%
%     LINKS
%%%%%%%%%%%%%%%%%%%%%%%%%%%%%%%%%%%%%%

\section{Links}
GitHub:// \href{https://github.com/matteeyah}{\bf matteeyah \faExternalLink} \\
LinkedIn://  \href{https://www.linkedin.com/in/matteeyah}{\bf matteeyah \faExternalLink} \\
StackOverflow://  \href{https://stackoverflow.com/users/1139722/matteeyah}{\bf matteeyah \faExternalLink}

%%%%%%%%%%%%%%%%%%%%%%%%%%%%%%%%%%%%%%
%     Projects
%%%%%%%%%%%%%%%%%%%%%%%%%%%%%%%%%%%%%%

\section{Projects}
\subsection{Respondo}
\location{\href{https://respondohub.com/}{Website \faExternalLink}}
Ticketing system for social media posts.
\sectionsep

\subsection{Tanukidesk}
\location{\href{https://gitlab.com/gitlab-com/marketing/community-relations/community-advocacy/tanukidesk}{GitLab Project \faExternalLink}}
Bidirectional communication between Zendesk and Disqus / HackerNews.
\sectionsep

\subsection{This Resume}
\location{\href{https://github.com/matteeyah/resume}{GitHub Project \faExternalLink}}
This resume is written in LaTeX and open-source.

%%%%%%%%%%%%%%%%%%%%%%%%%%%%%%%%%%%%%%
%     EDUCATION
%%%%%%%%%%%%%%%%%%%%%%%%%%%%%%%%%%%%%%

\section{Education}

\subsection{Metropolitan University}
\descript{BS in Software Engineering}
\location{October 2016 - | Belgrade, Serbia}
\vspace{\topsep}
\begin{tightemize}
\item Admitted on a full scholarship
\item Teaching assistant (Software Engineering and Distributed Systems)
\end{tightemize}
\sectionsep

\subsection{Electrotechnical School}
\location{Grad. June 2016 | Belgrade, Serbia}
\vspace{\topsep}
\begin{tightemize}
\item Student of the generation
\end{tightemize}
\sectionsep

%%%%%%%%%%%%%%%%%%%%%%%%%%%%%%%%%%%%%%
%     COURSEWORK
%%%%%%%%%%%%%%%%%%%%%%%%%%%%%%%%%%%%%%

\section{Coursework}
\subsection{Undergraduate}
Software Engineering \\
Systems Engineering \\
Hardware Systems \\
Distributed Systems \\
Concurrent Programming \\
Advanced Math \\
Machine Learning \\
\sectionsep

%%%%%%%%%%%%%%%%%%%%%%%%%%%%%%%%%%%%%%
%
%     COLUMN TWO
%
%%%%%%%%%%%%%%%%%%%%%%%%%%%%%%%%%%%%%%

\end{minipage}
\hfill
\begin{minipage}[t]{0.66\textwidth}

%%%%%%%%%%%%%%%%%%%%%%%%%%%%%%%%%%%%%%
%     EXPERIENCE
%%%%%%%%%%%%%%%%%%%%%%%%%%%%%%%%%%%%%%

\section{Experience}
\runsubsection{GitLab}
\descript{| Backend Engineer }
\location{October 2018 - Present | Remote}
\vspace{\topsep}
Engineering software has always been my greatest passion.\\
\vspace{\topsep}
Over time I became a reviewer and maintainer for various GitLab components.\\
\vspace{\topsep}
I drove and implemented over 220 deliverables in the main web app and surrounding micro-services. Some notable work includes:
\vspace{\topsep}
\begin{tightemize}
\item Support for multiple Kubernetes clusters per project | \href{https://gitlab.com/gitlab-org/gitlab/-/issues/3734}{Issue \faicon{external-link}}
\item Mechanism for retaining latest artifact per ref indefinitely | \href{https://gitlab.com/gitlab-org/gitlab/-/issues/16267}{Issue \faicon{external-link}} | \href{https://gitlab.com/gitlab-org/gitlab/-/merge_requests/29802}{Merge Request \#1 \faicon{external-link}} | \href{https://gitlab.com/gitlab-org/gitlab/-/merge_requests/30741}{Merge Request \#2 \faicon{external-link}}
\end{tightemize}
\sectionsep

\runsubsection{GitLab}
\descript{| Community Advocate Manager }
\location{April 2018 – October 2018 | Remote}
\vspace{\topsep}
I always liked helping others do the best work they can. As a Community Advocate Manager I built a new advocacy team.
\vspace{\topsep}
\begin{tightemize}
\item Implemented the GitLab Ultimate for Education and Open Source programs
\item Hired and onboarded a whole new Community Advocacy team
\end{tightemize}
\sectionsep

\runsubsection{GitLab}
\descript{| Community Advocate }
\location{November 2016 – April 2018 | Remote}
\vspace{\topsep}
I started working as a Community Advocate at GitLab with a mission to cultivate and reshape GitLab's online community.
\vspace{\topsep}
\begin{tightemize}
\item Implemented custom middleware that integrates Disqus and HackerNews into Zendesk
\item Migrated the Community Advocacy team from a shared Zendesk instance
\item Reduced friction in the Community Advocacy workflow
\item Brought all response time averages below 6h
\end{tightemize}
\sectionsep

%%%%%%%%%%%%%%%%%%%%%%%%%%%%%%%%%%%%%%
%     AWARDS
%%%%%%%%%%%%%%%%%%%%%%%%%%%%%%%%%%%%%%

\section{Awards (Computer Science)}
\begin{tabular}{rll}
2016 & 1\textsuperscript{st} / Country & IT Project of the Year\\
2016 & Gold Medal / World & International Conference of Young Scientists\\
2016 & 1\textsuperscript{st} / Country & National conference of research projects\\
2015 & 3\textsuperscript{rd} / Country & National conference of research projects\\
2014 & 1\textsuperscript{st} / Country & National conference of research projects\\
2014 & 3\textsuperscript{rd} / Country & National conference of research project
\end{tabular}
\sectionsep

%%%%%%%%%%%%%%%%%%%%%%%%%%%%%%%%%%%%%%
%     PUBLICATIONS
%%%%%%%%%%%%%%%%%%%%%%%%%%%%%%%%%%%%%%

\section{Publications}
\renewcommand\refname{\vskip -1.5em} % Couldn't get this working from the .cls file
\bibliographystyle{abbrv}
\bibliography{publications}
\nocite{*}

\end{minipage}
\end{document}  \documentclass[]{article}
